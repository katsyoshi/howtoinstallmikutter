\documentclass{jsarticle}
\usepackage{listings}
\pagestyle{empty}
\title{How to Install Mikutter?}
\author{@katsyoshi}
\date{}
\begin{document}
\maketitle
 \section{はじめに}
 この文書では,mikutter\cite{mikutter}をインストールし実行する方法につい
 て書いてあります.ここでは,Linux(\ref{linux}), FreeBSD(\ref{freebsd}),
 Mac(\ref{mac})におけるインストール方法について述べます.
 
 \section{インストール} \label{install}
 インストール方法についてある程度まで実行できるようにします.
  \subsection{必須アプリケーション}
  mikutterを動かすためには,以下のアプリケーションが必要となります\cite{readme}.
  \begin{itemize}
   \item ruby-1.9.2 以上
   \item x11
   \item gtk2
   \item Open SSL
   \item Subversion
  \end{itemize}
  これらのアプリケーションはmikutterをインストールする前にインストールし
  ます.mikutterを動かすために必要ではないが,subversionをあげておく.こ
  の文書でmikutterは,基本的にtrunk版を推奨する.

  Rubyはパッケージからでもインストールできますが,提供されてるRuby
  のパッケージが1.8系となっているディストリビューションも多いので,ここ
  では,RVM(Ruby Version Manager)について\ref{RVM}で述べます.
  
  \subsubsection{RVM:Ruby Version Manager} \label{RVM}
  mikutterをインストールする前にRubyの実行環境をRVM(Ruby Version
  Manager)\cite{rvm}を用いて構築します.
  Linuxの一部ディストリビューション,FreeBSDでは必要ありませんが,ここで
  は,一応説明します.
  RVMの公式サイトを参考にRVMをインストールし,Rubyの最新版をインストール
  します.
  \begin{lstlisting}
   $ bash -s stable < <(curl -s \
   https://raw.github.com/wayneeseguin/rvm/master/binscripts/rvm-installer)
   $ rvm install 1.9.3 --default
  \end{lstlisting}
  RVMをインストールするには,curl, gccなどが必要になりますのでRVM
  の公式サイトを参考にインストールしてください.
  
  \subsection{Linux} \label{linux}
  まず,Linuxでのインストール方法について説明します.Linuxでは,Ubuntu
  Linux(\ref{ubuntu})とGentoo Linux(\ref{gentoo})について説明します.

   \subsubsection{Ubuntu Linux} \label{ubuntu}
   Ubuntuでは,READMEの通りにいけますが,ここでは,RVMを用いてインストー
   ルします.そのまえにまず,必要なアプリケーションをインストールします.
    \begin{lstlisting}
     $ sudo apt-get update
     $ sudo apt-get upgrade
     $ sudo apt-get build-dep ruby
     $ sudo apt-get install libgtk2.0-dev
    \end{lstlisting}
    必要なアプリケーションをインストールする前にOSを最新の状態にします.
    つぎに,Rubyをビルドするために必要なアプリケーションをインストールし
    ます.そして,libgtk2.0-devをインストールします.これは,gemから
    ruby-gtk2をインストールするために必要です.

    RVMをインストールしてください(\ref{RVM}).RVMをインストー
    ル,ruby-1.9.3のインストールが完了したら,つぎにgemから必要なライブ
    ラリをインストールします.
    \begin{lstlisting}
     $ gem install gtk2
    \end{lstlisting}
    これで必要なものがインストールできました.

   \subsubsection{Gentoo Linux} \label{gentoo}
   Gentooでは,すでにパッケージが用意されてますので,コマンドを入力する
   とインストールできます.
   \begin{lstlisting}
    $ emerge mikutter
   \end{lstlisting}
   でインストール終了です.
  \subsection{FreeBSD} \label{freebsd}
  FreeBSDもGentoo(\ref{gentoo})と同様にportsが提供されていますので,簡単に
  インストールできます.
  \begin{lstlisting}
   $ cd /usr/ports/net-im/mikutter
   # make install
  \end{lstlisting}
  でインストール終了です.
  \subsection{Mac} \label{mac}
  Macでは,MacportsかHomebrewを用いて必要なライブラリをインストールしま
  す.ここでは,HomebrewとRVMを用いた方法について述べます.

  まず,Homebrewから必要なライブラリをインストールします.
  \begin{lstlisting}
   $ brew install gtk subversion
   $ gem install gtk2
  \end{lstlisting}
  準備はこれで終わりですが,日本語入力用にMacUIMをインストールする必要\ref{uim}が
  あります.

  \subsubsection{MacUIM} \label{uim}
  MacUIM\cite{macuim}は,MacのX11で日本語入力するためのアプリケーションです.
  MacUIMのインストールは,公式サイトから最新版のdmgをダウンロードします.

  公式サイトからダウンロードし,インストールが完了すると,つぎにMacUIMの
  設定を行ないます.MacUIMをX11でも使えるよう\cite{rubyneco}にまず,X11の
  /usr/X11/lib/X11/xinit/xinitrcを\~/.xinitrcへコピーします.
  コピーしたら,つぎの一行を追加します.
  \begin{lstlisting}
   /Library/Frameworks/UIM.framework/Versions/Current/bin/uim-xim &
  \end{lstlisting}
  これは,uimをバックグラウンドで起動します.つぎに,.MacOSディレクトリ
  を作成し,以下の内容でenvironment.plistを作成します.
  \begin{lstlisting}
   <?xml version="1.0" encoding="UTF-8"?>
   <!DOCTYPE plist PUBLIC "-//Apple Computer//DTD PLIST 1.0//EN"
   "http://www.apple.com/DTDs/PropertyList-1.0.dtd">
   <plist version="1.0">
   <dict>
   <key>LANG</key>
   <string>ja_JP.UTF-8</string>
   <key>GTK_IM_MODULE</key>
   <string>uim</string>
   <key>XMODIFIERS</key>
   <string>@im=uim</string>
   </dict>
   </plist>
  \end{lstlisting}
  最後にX11起動時にUIMが起動するように
  /usr/X11/lib/X11/xinit/xinitrc.d/20-uim.shを作成します.
  \begin{lstlisting}
   IM=/Library/Frameworks/UIM.framework/Versions/Current/bin/uim-xim
   [ -x $IM ] && $IM &
  \end{lstlisting}
  これでMacでの日本語入力が可能となります.
  
  \subsection{mikutterのダウンロード} \label{download}
  Gentoo LinuxやFreeBSDを利用していない場合は,以下のSVNを用いて最新版
  (開発版)をダウンロードします.
  \begin{lstlisting}
   $ svn checkout svn://toshia.dip.jp/mikutter/trunk mikutter
  \end{lstlisting}
 \section{実行} \label{execute}
 実行は,インストールした方法により変わってきます.Gentoo LinuxやFreeBSD
 では,コマンドラインにmikutterと入力するだけで実行できます.RVMを用いた
 場合では,
 \begin{lstlisting}
  $ cd mikutter
  $ ruby mikutter.rb
 \end{lstlisting}
 と入力することで実行することができます.
 \section{まとめ}
 mikutterのインストール方法について,簡単ですが述べてきました.mikutter
 を動作させるために,rubyなどの必要なアプリケーションのインストール,RVM
 を用いたRubyのインストールについて説明をし,OSごとのインストール方法に
 ついて書きました.
 
 つたない説明で分かりずらいと思ますので,Twitterで@katsyoshiにリプライを
 投げてくださると反応すると思います.

 \begin{thebibliography}{}
  \bibitem{mikutter} mikutter: http://mikutter.hachune.net
  \bibitem{readme} README: https://github.com/katsyoshi/mikutter
  \bibitem{rvm} RVM: http://beginrescueend.com/
  \bibitem{macuim} MacUIM: http://code.google.com/p/macuim/
  \bibitem{rubyneco} 猫にルビー X11上の日本語入力環境を構築する: http://catruby.blog83.fc2.com/blog-entry-10.html
 \end{thebibliography}
\end{document}